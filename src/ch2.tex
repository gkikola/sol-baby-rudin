\chapter{Basic Topology}

\Exercise1 Prove that the empty set is a subset of every set.
\begin{proof}
  Let $A$ be any set. Since the empty set has no elements, it is
  vacuously true that for every $x$ in the empty set, $x\in A$.
\end{proof}

\Exercise2 A complex number $z$ is said to be {\em algebraic} if there
are integers $a_0,\dots,a_n$, not all zero, such that
\begin{equation}
  \label{eq:complex-polynomial}
  a_0z^n + a_1z^{n-1} + \cdots + a_{n-1}z + a_n = 0.
\end{equation}
Prove that the set of all algebraic numbers is countable.
\begin{proof}
  For each positive integer $N$, let $E_N$ denote the set of all
  algebraic numbers $z$ satisfying \eqref{eq:complex-polynomial} where
  \begin{equation*}
    n + \abs{a_0} + \abs{a_1} + \cdots + \abs{a_n} = N.
  \end{equation*}
  Since all the terms on the left-hand side are positive integers, it
  follows that for each $N$ there are only finitely many such
  polynomial equations.  And for any fixed $n\leq N$, any polynomial
  of degree $n$ has a finite number of roots. Hence the set $E_N$ is
  finite.

  If $A$ denotes the set of all algebraic numbers, then we have
  \begin{equation*}
    A = \bigcup_{N=1}^\infty E_N.
  \end{equation*}
  Being the union of countably many finite sets, $A$ must be at most
  countable by the corollary to Theorem~2.12. And since $A$ must be
  infinite (for example, $Z$ is a subset) this shows that $A$ is
  countable.
\end{proof}

\Exercise3 Prove that there exist real numbers which are not
algebraic.
\begin{proof}
  By the previous exercise, we know that the set $A$ of algebraic
  numbers is countable. If $A = R$, then $R$ is countable, so $A$ must
  be a proper subset of $R$. Thus we can find an $x\in R$ with
  $x\not\in A$.
\end{proof}

\Exercise4 Is the set of all irrational real numbers countable?
\begin{solution}
  Suppose that the irrationals $R-Q$ are countable. Since
  \begin{equation*}
    R = Q\cup(R-Q),
  \end{equation*}
  this means that $R$ is the union of countable sets and is therefore
  countable by Theorem~2.12. This contradiction shows that $R-Q$ is
  uncountable.
\end{solution}

\Exercise5 Construct a bounded set of real numbers with exactly three
limit points.
\begin{solution}
  For each integer $m$, consider the set
  \begin{equation*}
    E_m = \left\{ m + \frac1{n+1}
      \;\middle|\;
      n\in Z^+ \right\},
  \end{equation*}
  where $Z^+$ denotes the positive integers. Then the set
  $A = E_0\cup E_1\cup E_2$ has exactly three limit points, namely the
  points $0$, $1$, and $2$, which we will now demonstrate. First note
  that any neighborhood of $0$ must contain points in $E_0$, and
  similarly for $1$ and $2$, so that $0,1,2\in A'$.

  On the other hand, suppose $x\in R-\{0,1,2\}$. Let
  \begin{equation*}
    r = \min\{\abs{x}, \abs{1 - x}, \abs{2 - x}, \abs{3 - x}\}.
  \end{equation*}
  Then the interval $(x - r/2, x + r/2)$ contains finitely many points
  in $A$. So by Theorem~2.20, it follows that $x$ is not a limit point
  of $A$.

  Therefore, the only limit points of $A$ are $0$, $1$, and $2$.
\end{solution}

\Exercise6 Let $E'$ be the set of all limit points of a set $E$. Prove
that $E'$ is closed. Prove that $E$ and $\overline{E}$ have the same limit
points. Do $E$ and $E'$ always have the same limit points?
\begin{proof}
  First we will show that any limit point of $\overline{E}$ is also a
  limit point of $E$. Let $x$ be a limit point of $\overline{E}$ and
  let $r$ be any positive real number.

  We want to show that the neighborhood $N_r(x)$ must contain a point
  in $E$ distinct from $x$. But we know that $N_r(x)$ contains a point
  $y\in\overline{E}$ with $y\neq x$. So $y\in E$ or $y\in E'$. If
  $y\in E$ then we are done, so suppose $y\in E'$ but $y\not\in
  E$. Then $y$ is a limit point of $E$, so every neighborhood of $y$
  must contain a point in $E$. In particular, let
  \begin{equation*}
    s = \frac{r - d(x,y)}2,
  \end{equation*}
  and choose $z\in N_s(y)$ such that $z\in E$. Then since
  $N_s(y)\subset N_r(x)$, we have $z\in N_r(x)$ and $z\in E$. And
  $z\neq x$ since $x\not\in N_s(y)$. So $x$ is a limit point of $E$.

  Now we show the converse. Let $x$ be a limit point of $E$. Then
  every neighborhood of $x$ contains a point in $E$ distinct from $x$,
  but this point must also be in $\overline{E} = E\cup E'$. Therefore
  $x$ is a limit point of $\overline{E}$.

  We have shown that $E$ and $\overline{E}$ have exactly the same set
  of limit points. That is, $E' = (\overline{E})'$.

  Next, to show that $E'$ is closed, let $x$ be a limit point of
  $E'$. Then $x\in\overline{E}$. But $\overline{E}$ is closed by
  Theorem~2.27, so $\overline{E} = (\overline{E})'$. Therefore
  $x\in(\overline{E})' = E'$. Thus every limit point of $E'$ is in
  $E'$, so the set $E'$ is closed.

  Lastly, it is not the case that $E$ and $E'$ must have the same
  limit points. For example, take $E = \{ 1/n \mid n\in Z^+ \}$, where
  $Z^+$ denotes the positive integers. Then $E' = \{0\}$ but $(E')'$
  is the empty set.
\end{proof}

\Exercise7 Let $A_1,A_2,A_3,\dots$ be subsets of a metric space.
\begin{enumerate}
\item If $B_n = \bigcup_{i=1}^n A_i$, prove that
  $\overline{B_n} = \bigcup_{i=1}^n \overline{A_i}$, for
  $n = 1,2,3,\dots$.
\item If $B = \bigcup_{i=1}^\infty A_i$, prove that
  $\overline{B}\supset\bigcup_{i=1}^\infty\overline{A_i}$.
\end{enumerate}
Show, by an example, that this inclusion can be proper.
\begin{solution}
  \begin{enumerate}
  \item Let $B_n$ be as stated, and suppose $x\in\overline{B_n}$. Then
    either $x\in B_n$ or $x\in B_n'$. First, if $x\in B_n$, then
    $x\in A_i$ for some index $i$ and we have $x\in\overline{A_i}$ so
    that $x\in\bigcup_i\overline{A_i}$. Now suppose instead that
    $x\in B_n'$. We want to show that $x\in\overline{A_i}$ for some
    $i$. Let $N_r(x)$ be any neighborhood of $x$, and choose a point
    $y\neq x$ in this neighborhood such that $y\in B_n$ (this is
    possible since $x$ is a limit point of $B_n$). Then $y\in A_i$ for
    some index $i$, and such a $y$ can be found for any neighborhood
    of $x$, so $x$ is a limit point of $A_i$. That is,
    $x\in\overline{A_i}$. This shows that
    \begin{equation}
      \label{eq:closure-subset-union}
      \overline{B_n}\subset\bigcup_{i=1}^n\overline{A_i}.
    \end{equation}

    Next, suppose $x\in\bigcup_i\overline{A_i}$, so that
    $x\in\overline{A_i}$ for some index $i$. Then $x\in A_i$ or
    $x\in A_i'$. If $x\in A_i$ then $x\in\overline{B_n}$ and we are
    done. So suppose $x\in A_i'$. Let $N_r(x)$ be any neighborhood of
    $x$, and choose $y\neq x$ such that $y\in N_r(x)\cap A_i$. Then
    $y\in B_n$, which proves that $x\in\overline{B_n}$. This shows that
    \begin{equation}
      \label{eq:closure-supset-union}
      \overline{B_n}\supset\bigcup_{i=1}^n\overline{A_i}.
    \end{equation}
    Together, \eqref{eq:closure-subset-union} and
    \eqref{eq:closure-supset-union} show that
    $\overline{B_n} = \bigcup_i\overline{A_i}$. \qedhere

  \item Suppose $x\in\bigcup_i\overline{A_i}$. Then there is a
    positive integer $i$ such that $x\in\overline{A_i}$. This implies
    that $x\in A_i$ or $x\in A_i'$. If $x\in A_i$, then $x\in B$ so
    certainly $x\in\overline{B}$. On the other hand, if $x\in A_i'$,
    then for any neighborhood $N_r(x)$, we may find $y\neq x$ in this
    neighborhood such that $y\in A_i$. Then $y\in B$, which proves
    that $x$ is a limit of point of $B$. So in either case,
    $x\in\overline{B}$, and the inclusion
    $\overline{B}\supset\bigcup_i\overline{A_i}$ is proved.

    Lastly, we show that this inclusion can be proper. For each
    positive integer $i$, let $A_i = \{1 / i\}$. That is, let each
    $A_i$ contain only one point, namely the reciprocal of the
    index. Then each $\overline{A_i}$ also consists of only this one
    point, so $\bigcup_{i=1}^\infty\overline{A_i}$ is the set
    $\{1/i \mid i\in Z^+\}$. However, $\overline{B}$ contains the
    point $0$, which is not in $\bigcup\overline{A_i}$. \qed
  \end{enumerate}
\end{solution}

\Exercise8 Is every point of every open set $E\subset R^2$ a limit
point of $E$? Answer the same question for closed sets in $R^2$.
\begin{solution}
  We will show that every point of every open set $E$ in $R^2$ is a
  limit point of $E$. Let $\vec{x}\in E$. Since $E$ is open, $\vec{x}$
  is an interior point, and we can find a neighborhood
  $N_r(\vec{x})\subset E$. Since $E\subset R^2$, there are infinitely
  many points in $N_r(\vec{x})$ distinct from $\vec{x}$, and this is
  still true if we use a smaller positive value for $r$. Therefore
  every neighborhood of $\vec{x}$ contains a point in $E$ distinct
  from $\vec{x}$. This means that $\vec{x}$ is a limit point of
  $E$. It follows that every point in $E$ is a limit point of $E$.

  The same is not true for closed sets in $R^2$. For example, the set
  \begin{equation*}
    E = \{(0,0)\}\cup\{(1/n,0)\in R^2\mid n\in Z^+\}
  \end{equation*}
  contains its only limit point $(0,0)$ and is thus closed. However,
  $(1, 0)\in E$ but $(1,0)\not\in E'$.
\end{solution}
