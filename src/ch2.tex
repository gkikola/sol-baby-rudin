\chapter{Basic Topology}

\Exercise1 Prove that the empty set is a subset of every set.
\begin{proof}
  Let $A$ be any set. Since the empty set has no elements, it is
  vacuously true that for every $x$ in the empty set, $x\in A$.
\end{proof}

\Exercise2 A complex number $z$ is said to be {\em algebraic} if there
are integers $a_0,\dots,a_n$, not all zero, such that
\begin{equation}
  \label{eq:complex-polynomial}
  a_0z^n + a_1z^{n-1} + \cdots + a_{n-1}z + a_n = 0.
\end{equation}
Prove that the set of all algebraic numbers is countable.
\begin{proof}
  For each positive integer $N$, let $E_N$ denote the set of all
  algebraic numbers $z$ satisfying \eqref{eq:complex-polynomial} where
  \begin{equation*}
    n + \abs{a_0} + \abs{a_1} + \cdots + \abs{a_n} = N.
  \end{equation*}
  Since all the terms on the left-hand side are positive integers, it
  follows that for each $N$ there are only finitely many such
  polynomial equations.  And for any fixed $n\leq N$, any polynomial
  of degree $n$ has a finite number of roots. Hence the set $E_N$ is
  finite.

  If $A$ denotes the set of all algebraic numbers, then we have
  \begin{equation*}
    A = \bigcup_{N=1}^\infty E_N.
  \end{equation*}
  Being the union of countably many finite sets, $A$ must be at most
  countable by the corollary to Theorem~2.12. And since $A$ must be
  infinite (for example, $Z$ is a subset) this shows that $A$ is
  countable.
\end{proof}

\Exercise3 Prove that there exist real numbers which are not
algebraic.
\begin{proof}
  By the previous exercise, we know that the set $A$ of algebraic
  numbers is countable. If $A = R$, then $R$ is countable, so $A$ must
  be a proper subset of $R$. Thus we can find an $x\in R$ with
  $x\not\in A$.
\end{proof}

\Exercise4 Is the set of all irrational real numbers countable?
\begin{solution}
  Suppose that the irrationals $R-Q$ are countable. Since
  \begin{equation*}
    R = Q\cup(R-Q),
  \end{equation*}
  this means that $R$ is the union of countable sets and is therefore
  countable by Theorem~2.12. This contradiction shows that $R-Q$ is
  uncountable.
\end{solution}

\Exercise5 Construct a bounded set of real numbers with exactly three
limit points.
\begin{solution}
  For each integer $m$, consider the set
  \begin{equation*}
    E_m = \left\{ m + \frac1{n+1}
      \;\middle|\;
      n\in Z^+ \right\},
  \end{equation*}
  where $Z^+$ denotes the positive integers. Then the set
  $A = E_0\cup E_1\cup E_2$ has exactly three limit points, namely the
  points $0$, $1$, and $2$, which we will now demonstrate. First note
  that any neighborhood of $0$ must contain points in $E_0$, and
  similarly for $1$ and $2$, so that $0,1,2\in A'$.

  On the other hand, suppose $x\in R-\{0,1,2\}$. Let
  \begin{equation*}
    r = \min\{\abs{x}, \abs{1 - x}, \abs{2 - x}, \abs{3 - x}\}.
  \end{equation*}
  Then the interval $(x - r/2, x + r/2)$ contains finitely many points
  in $A$. So by Theorem~2.20, it follows that $x$ is not a limit point
  of $A$.

  Therefore, the only limit points of $A$ are $0$, $1$, and $2$.
\end{solution}
