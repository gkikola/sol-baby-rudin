\chapter{Differentiation}

\Exercise1 Let $f$ be defined for all real $x$, and suppose that
\begin{equation}
  \label{exercise:diff:f-bounded-by-x-minus-y-sq}
  \abs{f(x) - f(y)} \leq (x - y)^2
\end{equation}
for all real $x$ and $y$. Prove that $f$ is constant.
\begin{proof}
  From \eqref{exercise:diff:f-bounded-by-x-minus-y-sq}, if $x\neq y$
  we have
  \begin{equation*}
    \Abs{\frac{f(x) - f(y)}{x-y}} \leq \abs{x - y}.
  \end{equation*}
  Now letting $y\to x$ we get
  \begin{equation*}
    f'(x) = 0.
  \end{equation*}
  Since this is true for all $x$, $f$ must be constant by
  Theorem~5.11.
\end{proof}

\Exercise2 Suppose $f'(x) > 0$ in $(a,b)$. Prove that $f$ is strictly
increasing in $(a,b)$, and let $g$ be its inverse function. Prove that
$g$ is differentiable, and that
\begin{equation*}
  g'(f(x))
  = \frac1{f'(x)}
  \quad
  (a < x < b).
\end{equation*}
\begin{proof}
  First, since $f'(x) > 0$, we know by Theorem~5.11 that $f$ is
  monotonically increasing in $(a,b)$. Suppose $f(x) = f(y)$ for
  $a<x<y<b$. By the mean value theorem, there is a $c\in(x,y)$ such
  that $f'(c) = 0$, but this is a contradiction. Therefore $f$ is
  strictly increasing in $(a,b)$.

  Since $f$ is strictly increasing, it is a one-to-one function and
  the inverse $g$ exists. Moreover, since the restriction of $f$ to
  any interval $[c,d]$ such that $a < c < d < b$ is a continuous
  mapping on a compact metric space, the inverse $g$ must also be
  continuous on $f([c,d])$. Since $c$ and $d$ were arbitrary, $g$ must
  be continuous on $(a,b)$.

  Let $\varepsilon > 0$. Then we may choose $\eta > 0$ such that
  $0 < \abs{x - t} < \eta$ implies
  \begin{equation*}
    \Abs{\frac1{f'(x)} - \frac{x - t}{f(x) - f(t)}} < \varepsilon.
  \end{equation*}
  Since $g$ is continuous, we may choose $\delta > 0$ so that
  $0 < \abs{p - q} < \delta$ implies $\abs{g(p) - g(q)} < \eta$. Now
  pick any $p,q\in I$ with $p\neq q$ such that $\abs{p - q} < \delta$,
  and set $x = g(p)$ and $t = g(q)$. Then $0 < \abs{x - t} < \eta$ and
  we have
  \begin{equation*}
    \Abs{\frac1{f'(x)} - \frac{g(p) - g(q)}{p - q}}
    = \Abs{\frac1{f'(x)} - \frac{x - t}{f(x) - f(t)}}
    < \varepsilon.
  \end{equation*}
  It now follows that
  \begin{equation*}
    g'(f(x)) = \lim_{q\to p}\frac{g(p) - g(q)}{p - q}
    = \frac1{f'(x)}. \qedhere
  \end{equation*}
\end{proof}

\Exercise3 Suppose $g$ is a real function on $R^1$, with bounded
derivative (say $\abs{g'} \leq M$). Fix $\varepsilon > 0$, and define
$f(x) = x + \varepsilon g(x)$. Prove that $f$ is one-to-one if
$\varepsilon$ is small enough. (A set of admissible values of
$\varepsilon$ can be determined which depends only on $M$.)
\begin{proof}
  If $0 < \varepsilon < 1/M$, then
  \begin{equation*}
    f'(x) = 1 + \varepsilon g'(x)
    \geq 1 - \varepsilon M > 0.
  \end{equation*}
  By the previous exercise, we know that $f$ is one-to-one.
\end{proof}
