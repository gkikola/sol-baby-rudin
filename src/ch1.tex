\chapter{The Real and Complex Number Systems}

Unless explicitly stated otherwise, all numbers in this chapter's
exercises are understood to be real.

\Exercise1 If $r$ is rational ($r\neq0$) and $x$ is irrational, prove
that $r + x$ and $rx$ are irrational.
\begin{proof}
  Suppose $r$ is rational, so that $r = a/b$ for $a,b\in Z$, and let
  $x$ be irrational. Now assume for contradiction that $r + x$ is
  rational. Then, for some integers $c$ and $d$, we can write
  \begin{equation*}
    x = (r + x) - r = \frac{c}{d} - \frac{a}{b} = \frac{bc - ad}{bd},
  \end{equation*}
  so that $x$ is rational: a contradiction. Therefore $r + x$ must be
  irrational.

  Similarly, if we assume that $rx = c/d$ for $c,d\in Z$, then, since
  $r\neq0$,
  \begin{equation*}
    x = \frac1r(rx) = \frac{b}{a}\cdot\frac{c}{d} = \frac{bc}{ad},
  \end{equation*}
  again a contradiction. Hence $rx$ must be irrational.
\end{proof}

\Exercise2 Prove that there is no rational number whose square is
$12$.
\begin{proof}
  Assume the contrary, and let $p$ and $q$ be integers such that
  $p^2 = 12q^2$. We may further suppose that $p$ and $q$ have no
  common factors other than $1$. Then $2$ divides $p^2$ so that $2$
  divides $p$ as well. Thus we can write $p = 2k$ for $k\in Z$. Then
  $4k^2 = 12q^2$, which implies that $k^2 = 3q^2$.

  Now $3$ divides $k^2$, so $3$ divides $k$ as well (if $3$ does not
  divide $k$, then $3$ could not divide $k^2$ since $3$ is prime),
  allowing us to write $k = 3\ell$ for $\ell\in Z$. Hence
  $3q^2 = 9\ell^2$ which implies that $q^2 = 3\ell^2$. Then $3$
  divides $q^2$ and thus $q$ as well. Since $3$ divides $k$, it must
  divide $p$, and we see that $3$ is a common factor of $p$ and $q$,
  which contradicts our choice of $p$ and $q$. Therefore there is no
  rational number $p/q$ whose square is $12$.
\end{proof}

\Exercise3 Prove Proposition~1.15:
\begin{prop}
  Let $F$ be a field and let $x,y,z\in F$. Then the following
  properties hold.
  \begin{enumerate}
  \item If $x\neq0$ and $xy = xz$ then $y = z$.
  \item If $x\neq0$ and $xy = x$ then $y = 1$.
  \item If $x\neq0$ and $xy = 1$, then $y = 1/x$.
  \item If $x\neq0$ then $1/(1/x) = x$.
  \end{enumerate}
\end{prop}
\begin{proof}
  \begin{enumerate}
  \item Let $x,y,z$ be such that $xy = xz$, with $x\neq0$. Since $x$
    is nonzero, it has a multiplicative inverse $1/x$. So
    \begin{equation*}
      y = 1y = \left(\frac1x\cdot x\right)y
      = \frac1x(xy) = \frac1x(xz)
      = \left(\frac1x\cdot x\right)z = 1z = z.
    \end{equation*}

  \item Suppose $xy = x$, $x\neq0$. As before, $1/x$ exists, so we have
    \begin{equation*}
      y = 1y = \left(\frac1x\cdot x\right)y
      = \frac1x(xy) = \frac1x\cdot x = 1.
    \end{equation*}

  \item \label{y-eq-recip-x-proof} If $xy = 1$, $x\neq0$, then
    \begin{equation*}
      y = 1y = \left(\frac1x\cdot x\right)y
      = \frac1x(xy) = \frac1x\cdot1
      = \frac1x.
    \end{equation*}

  \item Let $x\neq0$. Since $(1/x)x = 1$, it follows from part
    \ref{y-eq-recip-x-proof} above that $x = 1/(1/x)$.
  \end{enumerate}
\end{proof}

\Exercise4 Let $E$ be a nonempty subset of an ordered set; suppose
$\alpha$ is a lower bound of $E$ and $\beta$ is an upper bound of
$E$. Prove that $\alpha\leq\beta$.
\begin{proof}
  Since $E$ is nonempty, we may choose $n\in E$. $\alpha$ is a lower
  bound of $E$, so $\alpha\leq n$. On the other hand, $\beta$ is an
  upper bound, so $n\leq\beta$. By the transitive property for ordered
  sets (see Definition~1.5 in the text), $\alpha\leq n$ and
  $n\leq\beta$ together imply that $\alpha\leq\beta$.
\end{proof}

\Exercise5 Let $A$ be a nonempty set of real numbers which is bounded
below. Let $-A$ be the set of all numbers $-x$, where $x\in A$. Prove
that
\begin{equation*}
  \inf A = -\sup(-A).
\end{equation*}
\begin{proof}
  Let $a = \inf A$ and choose any $b\in-A$ (we know $-A$ is nonempty
  since $A$ is nonempty). Then $-b\in A$ so that $a\leq-b$, which
  implies $b\leq-a$. And $b$ was arbitrary, so $-a = -\inf A$ is an upper bound
  for $-A$.

  $-A$ is bounded above, so it has a least upper bound $c =
  \sup(-A)$. We need to show that $c = -a$. Assume the contrary, and
  suppose $c < -a$. Then since $c$ is an upper bound for $-A$, we have
  $c\geq b$ for all $b$ in $-A$. Then $-c\leq-b$, with $-b\in A$, so
  that $-c$ is a lower bound for $A$. But $c < -a$, so $-c > a$. Hence
  $-c$ is a lower bound for $A$, and it is larger than $a = \inf A$, a
  contradiction. This shows that $c = -a$, so that
  $\sup(-A) = -\inf A$ as required.
\end{proof}

% \Exercise6 Fix $b>1$.
% \begin{enumerate}
% \item If $m, n, p, q$ are integers, $n>0, q>0$, and $r = m/n = p/q$,
%   prove that
%   \begin{equation}
%     \label{eq:ratb}
%     (b^m)^{1/n} = (b^p)^{1/q}.
%   \end{equation}
%   Hence it makes sense to define $b^r = (b^m)^{1/n}$.
%   \begin{proof}
%     If $r = 0$ then the result is obvious, so suppose $r\neq0$.

%     Let $c = (b^m)^{1/n}$ and let $d = (b^p)^{1/q}$. We want to show
%     that $c = d$.

%     By definition, $b^m = c^n$ and $b^p = d^q$. Then
%     $b^{mp} = (b^m)^p = c^{np}$. However, $np = mq$, so this implies
%     that $b^{mp} = c^{mq}$. On the other hand,
%     $b^{mp} = (b^p)^m = d^{mq}$. Hence $c^{mq} = d^{mq}$.

%     Now either $m > 0$ or $m < 0$. In the former case, $mq > 0$ so
%     that $c = d$ by the uniqueness condition in Theorem~1.21. On the
%     other hand, if $m < 0$, then $c^{-mq} = d^{-mq}$ and $c = d$ by
%     the same argument. Therefore equation \eqref{eq:ratb} holds.
%   \end{proof}

% \item Prove that $b^{r+s} = b^rb^s$ if $r$ and $s$ are rational.
%   \begin{proof}
%     Let $r = m/n$ and $s = p/q$, where $m,n,p,q$ are integers with
%     $n,q>0$. Then
%     \begin{equation*}
%       (b^{r+s})^{nq} = b^{mq+np} = b^{mq}b^{np}.
%     \end{equation*}
%     Then
%     \begin{equation*}
%       b^{r+s} = (b^{mq}b^{np})^{1/(nq)},
%     \end{equation*}
%     and by the corollary to Theorem~1.21, we have
%     \begin{equation*}
%       b^{r+s} = (b^{mq})^{1/(nq)}(b^{np})^{1/(nq)} = b^rb^s.\qedhere
%     \end{equation*}
%   \end{proof}

% \item If $x$ is real, define $B(x)$ to be the set of all numbers
%   $b^t$, where $t$ is rational and $t\leq x$. Prove that
%   \begin{equation*}
%     b^r = \sup B(r)
%   \end{equation*}
%   when $r$ is rational. Hence it makes sense to define
%   \begin{equation*}
%     b^x = \sup B(x)
%   \end{equation*}
%   for every real $x$.
%   \begin{proof}
%     Let $r$ and $t$ be rational numbers, with $t\leq r$. Then there
%     are integers $m,n,p,q$ with $n,q>0$ such that $r = m/n$ and
%     $t = p/q$.

%     Let $\beta = b^{1/(nq)}$. We will show that $\beta$ must be
%     greater than $1$. For, if not, suppose $\beta\leq1$. Then
%     $\beta^2\leq\beta$, and similarly $\beta^3\leq\beta^2\leq\beta$,
%     and a simple induction argument will show that
%     $b = \beta^{nq}\leq\beta\leq1$. But $b > 1$, so this is a
%     contradiction. Therefore $\beta > 1$.

%     Now since $t\leq r$, we have $np\leq mq$, which implies that
%     $\beta^{np}\leq\beta^{mq}$. So we have
%     \begin{equation*}
%       b^t = (\beta^{nq})^t = \beta^{np}
%       \leq \beta^{mq} = (\beta^{nq})^r = b^r.
%     \end{equation*}
%     Since $t$ was arbitrary, $b^r$ is an upper bound for $B(r)$.

%     Finally, $b^r$ must be the least upper bound for $B(r)$, since
%     $b^r$ is itself a member of $B(r)$.
%   \end{proof}

% \item Prove that $b^{x+y} = b^xb^y$ for all real $x$ and $y$.
% \end{enumerate}

% \Exercise7 Fix $b>1, y>0$, and prove that there is a unique real $x$
% such that $b^x = y$, by completing the following outline. (This $x$ is
% called the {\em logarithm of $y$ to the base $b$}.)
% \begin{enumerate}
% \item For any positive integer $n$, $b^n-1\geq n(b - 1)$.
%   \label{b-inequality}
%   \begin{proof}
%     We use induction on $n$. For $n = 1$, the result is obvious. So
%     suppose it is true for $n = k$. Then, using this induction
%     hypothesis, we may write
%     \begin{align*}
%       b^{k+1} - 1
%       &= b^{k+1} - b^k + b^k - 1 \\
%       &\geq b^{k+1} - b^k + k(b - 1) \\
%       &= (b - 1)(b^k + k).
%     \end{align*}
%     And finally, since $b > 1$, it follows that $b^k>1$ which gives
%     \begin{equation*}
%       b^{k+1} - 1 \geq (k + 1)(b - 1),
%     \end{equation*}
%     which completes the inductive step. Hence the result is true for
%     all integers $n\geq1$.
%   \end{proof}
% \item Hence $b - 1\geq n(b^{1/n} - 1)$.
%   \label{b-nth-root-inequality}
%   \begin{proof}
%     Take the result from part \ref{b-inequality} and replace $b$ with
%     $b^{1/n}$, noting that $b^{1/n}$ must be greater than $1$.
%   \end{proof}
% \item If $t > 1$ and $n > (b - 1)/(t - 1)$, then $b^{1/n} < t$.
%   \label{b-nth-root-less-than-t}
%   \begin{proof}
%     Suppose $t$ and $n$ satisfy the stated conditions. Part
%     \ref{b-nth-root-inequality} implies
%     \begin{align*}
%       n > \frac{b-1}{t-1} \geq \frac{n(b^{1/n} - 1)}{t-1}
%     \end{align*}
%     or
%     \begin{equation*}
%       1 > \frac{b^{1/n} - 1}{t-1}.
%     \end{equation*}
%     Since $t>1$, we can rearrange to get
%     \begin{equation*}
%       t > b^{1/n}
%     \end{equation*}
%     as desired.
%   \end{proof}
% \item If $w$ is such that $b^w<y$, then $b^{w+(1/n)}<y$ for
%   sufficiently large $n$.
%   \begin{proof}
%     \label{b-to-the-w-less-than-y}
%     Let $t = yb^{-w}$. Since $y > b^w$, we have $t > 1$. By the
%     archimedean property of $R$ (see Theorem~1.20a), there is a
%     positive integer $n$ such that
%     \begin{equation*}
%       n(t - 1) > b - 1.
%     \end{equation*}
%     So, by part \ref{b-nth-root-less-than-t}, we have
%     $b^{1/n} < t = yb^{-w}$ which gives
%     \begin{equation*}
%       b^{w+(1/n)} < y
%     \end{equation*}
%     as required.
%   \end{proof}
% \item If $b^w > y$, then $b^{w-(1/n)}>y$ for sufficiently large $n$.
%   \begin{proof}
%     \label{b-to-the-w-greater-than-y}
%     Let $t = b^w/y$. Then $t > 1$. Again, we can find an $n$ such that
%     $n(t-1) > b-1$ so part \ref{b-nth-root-less-than-t} gives
%     $b^{1/n} < b^w/y$ or, rearranging,
%     \begin{equation*}
%       b^{w-(1/n)} > y. \qedhere
%     \end{equation*}
%   \end{proof}
% \item Let $A$ be the set of all $w$ such that $b^w < y$, and show that
%   $x = \sup A$ satisfies $b^x = y$.
%   \begin{proof}
%     There are three possibilities: $b^x < y$, $b^x > y$, or $b^x = y$.

%     Suppose that $b^x < y$. Then by part \ref{b-to-the-w-less-than-y}
%     there is a positive integer $n$ such that $b^{x+1/n} <
%     y$. Therefore $x + 1/n\in A$ which contradicts the fact that $x$
%     is an upper bound for $A$.

%     On the other hand, if $b^x > y$, then part
%     \ref{b-to-the-w-greater-than-y} allows us to find a positive
%     integer $n$ such that $b^{x-1/n} > y$. Then $x-1/n\not\in A$ so
%     $x-1/n$ is an upper bound for $A$. But this contradicts the fact
%     that $x$ is the least upper bound for $A$.

%     We see that the only remaining possibility is that $b^x = y$.
%   \end{proof}
% \item Prove that this $x$ is unique.
%   \begin{proof}
%     Suppose $x$ is such that $b^x = y$ and let $z > x$. Then
%     \begin{equation*}
%       b^z = b^{x+(z-x)} = b^xb^{z-x} > b^x = y,
%     \end{equation*}
%     so $b^z\neq y$. Now suppose instead that $z < x$. If $b^z = y$
%     then the previous argument shows that $b^x\neq y$, which is a
%     contradiction. Therefore $b^z\neq y$ in this case as well. So $x$
%     is unique.
%   \end{proof}
% \end{enumerate}
