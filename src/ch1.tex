\chapter{The Real and Complex Number Systems}

Unless explicitly stated otherwise, all numbers in this chapter's
exercises are understood to be real.

\Exercise1 If $r$ is rational ($r\neq0$) and $x$ is irrational, prove
that $r + x$ and $rx$ are irrational.
\begin{proof}
  Suppose $r$ is rational, so that $r = a/b$ for $a,b\in Z$, and let
  $x$ be irrational. Now assume for contradiction that $r + x$ is
  rational. Then, for some integers $c$ and $d$, we can write
  \begin{equation*}
    x = (r + x) - r = \frac{c}{d} - \frac{a}{b} = \frac{bc - ad}{bd},
  \end{equation*}
  so that $x$ is rational: a contradiction. Therefore $r + x$ must be
  irrational.

  Similarly, if we assume that $rx = c/d$ for $c,d\in Z$, then, since
  $r\neq0$,
  \begin{equation*}
    x = \frac1r(rx) = \frac{b}{a}\cdot\frac{c}{d} = \frac{bc}{ad},
  \end{equation*}
  again a contradiction. Hence $rx$ must be irrational.
\end{proof}

\Exercise2 Prove that there is no rational number whose square is
$12$.
\begin{proof}
  Assume the contrary, and let $p$ and $q$ be integers such that
  $p^2 = 12q^2$. We may further suppose that $p$ and $q$ have no
  common factors other than $1$. Then $2$ divides $p^2$ so that $2$
  divides $p$ as well. Thus we can write $p = 2k$ for $k\in Z$. Then
  $4k^2 = 12q^2$, which implies that $k^2 = 3q^2$.

  Now $3$ divides $k^2$, so $3$ divides $k$ as well (if $3$ does not
  divide $k$, then $3$ could not divide $k^2$ since $3$ is prime),
  allowing us to write $k = 3\ell$ for $\ell\in Z$. Hence
  $3q^2 = 9\ell^2$ which implies that $q^2 = 3\ell^2$. Then $3$
  divides $q^2$ and thus $q$ as well. Since $3$ divides $k$, it must
  divide $p$, and we see that $3$ is a common factor of $p$ and $q$,
  which contradicts our choice of $p$ and $q$. Therefore there is no
  rational number $p/q$ whose square is $12$.
\end{proof}

\Exercise3 Prove Proposition~1.15:
\begin{prop}
  Let $F$ be a field and let $x,y,z\in F$. Then the following
  properties hold.
  \begin{enumerate}
  \item If $x\neq0$ and $xy = xz$ then $y = z$.
  \item If $x\neq0$ and $xy = x$ then $y = 1$.
  \item If $x\neq0$ and $xy = 1$, then $y = 1/x$.
  \item If $x\neq0$ then $1/(1/x) = x$.
  \end{enumerate}
\end{prop}
\begin{proof}
  \begin{enumerate}
  \item Let $x,y,z$ be such that $xy = xz$, with $x\neq0$. Since $x$
    is nonzero, it has a multiplicative inverse $1/x$. So
    \begin{equation*}
      y = 1y = \left(\frac1x\cdot x\right)y
      = \frac1x(xy) = \frac1x(xz)
      = \left(\frac1x\cdot x\right)z = 1z = z.
    \end{equation*}

  \item Suppose $xy = x$, $x\neq0$. As before, $1/x$ exists, so we have
    \begin{equation*}
      y = 1y = \left(\frac1x\cdot x\right)y
      = \frac1x(xy) = \frac1x\cdot x = 1.
    \end{equation*}

  \item \label{y-eq-recip-x-proof} If $xy = 1$, $x\neq0$, then
    \begin{equation*}
      y = 1y = \left(\frac1x\cdot x\right)y
      = \frac1x(xy) = \frac1x\cdot1
      = \frac1x.
    \end{equation*}

  \item Let $x\neq0$. Since $(1/x)x = 1$, it follows from part
    \ref{y-eq-recip-x-proof} above that $x = 1/(1/x)$.
  \end{enumerate}
\end{proof}

\Exercise4 Let $E$ be a nonempty subset of an ordered set; suppose
$\alpha$ is a lower bound of $E$ and $\beta$ is an upper bound of
$E$. Prove that $\alpha\leq\beta$.
\begin{proof}
  Since $E$ is nonempty, we may choose $n\in E$. $\alpha$ is a lower
  bound of $E$, so $\alpha\leq n$. On the other hand, $\beta$ is an
  upper bound, so $n\leq\beta$. By the transitive property for ordered
  sets (see Definition~1.5 in the text), $\alpha\leq n$ and
  $n\leq\beta$ together imply that $\alpha\leq\beta$.
\end{proof}

\Exercise5 Let $A$ be a nonempty set of real numbers which is bounded
below. Let $-A$ be the set of all numbers $-x$, where $x\in A$. Prove
that
\begin{equation*}
  \inf A = -\sup(-A).
\end{equation*}
\begin{proof}
  Let $a = \inf A$ and choose any $b\in-A$ (we know $-A$ is nonempty
  since $A$ is nonempty). Then $-b\in A$ so that $a\leq-b$, which
  implies $b\leq-a$. And $b$ was arbitrary, so $-a = -\inf A$ is an upper bound
  for $-A$.

  $-A$ is bounded above, so it has a least upper bound $c =
  \sup(-A)$. We need to show that $c = -a$. Assume the contrary, and
  suppose $c < -a$. Then since $c$ is an upper bound for $-A$, we have
  $c\geq b$ for all $b$ in $-A$. Then $-c\leq-b$, with $-b\in A$, so
  that $-c$ is a lower bound for $A$. But $c < -a$, so $-c > a$. Hence
  $-c$ is a lower bound for $A$, and it is larger than $a = \inf A$, a
  contradiction. This shows that $c = -a$, so that
  $\sup(-A) = -\inf A$ as required.
\end{proof}

% \Exercise6 Fix $b>1$.
% \begin{enumerate}
% \item If $m, n, p, q$ are integers, $n>0, q>0$, and $r = m/n = p/q$,
%   prove that
%   \begin{equation}
%     \label{eq:ratb}
%     (b^m)^{1/n} = (b^p)^{1/q}.
%   \end{equation}
%   Hence it makes sense to define $b^r = (b^m)^{1/n}$.
%   \begin{proof}
%     If $r = 0$ then the result is obvious, so suppose $r\neq0$.

%     Let $c = (b^m)^{1/n}$ and let $d = (b^p)^{1/q}$. We want to show
%     that $c = d$.

%     By definition, $b^m = c^n$ and $b^p = d^q$. Then
%     $b^{mp} = (b^m)^p = c^{np}$. However, $np = mq$, so this implies
%     that $b^{mp} = c^{mq}$. On the other hand,
%     $b^{mp} = (b^p)^m = d^{mq}$. Hence $c^{mq} = d^{mq}$.

%     Now either $m > 0$ or $m < 0$. In the former case, $mq > 0$ so
%     that $c = d$ by the uniqueness condition in Theorem~1.21. On the
%     other hand, if $m < 0$, then $c^{-mq} = d^{-mq}$ and $c = d$ by
%     the same argument. Therefore equation \eqref{eq:ratb} holds.
%   \end{proof}

% \item Prove that $b^{r+s} = b^rb^s$ if $r$ and $s$ are rational.
%   \begin{proof}
%     Let $r = m/n$ and $s = p/q$, where $m,n,p,q$ are integers with
%     $n,q>0$. Then
%     \begin{equation*}
%       (b^{r+s})^{nq} = b^{mq+np} = b^{mq}b^{np}.
%     \end{equation*}
%     Then
%     \begin{equation*}
%       b^{r+s} = (b^{mq}b^{np})^{1/(nq)},
%     \end{equation*}
%     and by the corollary to Theorem~1.21, we have
%     \begin{equation*}
%       b^{r+s} = (b^{mq})^{1/(nq)}(b^{np})^{1/(nq)} = b^rb^s.\qedhere
%     \end{equation*}
%   \end{proof}

% \item If $x$ is real, define $B(x)$ to be the set of all numbers
%   $b^t$, where $t$ is rational and $t\leq x$. Prove that
%   \begin{equation*}
%     b^r = \sup B(r)
%   \end{equation*}
%   when $r$ is rational. Hence it makes sense to define
%   \begin{equation*}
%     b^x = \sup B(x)
%   \end{equation*}
%   for every real $x$.
%   \begin{proof}
%     Let $r$ and $t$ be rational numbers, with $t\leq r$. Then there
%     are integers $m,n,p,q$ with $n,q>0$ such that $r = m/n$ and
%     $t = p/q$.

%     Let $\beta = b^{1/(nq)}$. We will show that $\beta$ must be
%     greater than $1$. For, if not, suppose $\beta\leq1$. Then
%     $\beta^2\leq\beta$, and similarly $\beta^3\leq\beta^2\leq\beta$,
%     and a simple induction argument will show that
%     $b = \beta^{nq}\leq\beta\leq1$. But $b > 1$, so this is a
%     contradiction. Therefore $\beta > 1$.

%     Now since $t\leq r$, we have $np\leq mq$, which implies that
%     $\beta^{np}\leq\beta^{mq}$. So we have
%     \begin{equation*}
%       b^t = (\beta^{nq})^t = \beta^{np}
%       \leq \beta^{mq} = (\beta^{nq})^r = b^r.
%     \end{equation*}
%     Since $t$ was arbitrary, $b^r$ is an upper bound for $B(r)$.

%     Finally, $b^r$ must be the least upper bound for $B(r)$, since
%     $b^r$ is itself a member of $B(r)$.
%   \end{proof}

% \item Prove that $b^{x+y} = b^xb^y$ for all real $x$ and $y$.
% \end{enumerate}

% \Exercise7 Fix $b>1, y>0$, and prove that there is a unique real $x$
% such that $b^x = y$, by completing the following outline. (This $x$ is
% called the {\em logarithm of $y$ to the base $b$}.)
% \begin{enumerate}
% \item For any positive integer $n$, $b^n-1\geq n(b - 1)$.
%   \label{b-inequality}
%   \begin{proof}
%     We use induction on $n$. For $n = 1$, the result is obvious. So
%     suppose it is true for $n = k$. Then, using this induction
%     hypothesis, we may write
%     \begin{align*}
%       b^{k+1} - 1
%       &= b^{k+1} - b^k + b^k - 1 \\
%       &\geq b^{k+1} - b^k + k(b - 1) \\
%       &= (b - 1)(b^k + k).
%     \end{align*}
%     And finally, since $b > 1$, it follows that $b^k>1$ which gives
%     \begin{equation*}
%       b^{k+1} - 1 \geq (k + 1)(b - 1),
%     \end{equation*}
%     which completes the inductive step. Hence the result is true for
%     all integers $n\geq1$.
%   \end{proof}
% \item Hence $b - 1\geq n(b^{1/n} - 1)$.
%   \label{b-nth-root-inequality}
%   \begin{proof}
%     Take the result from part \ref{b-inequality} and replace $b$ with
%     $b^{1/n}$, noting that $b^{1/n}$ must be greater than $1$.
%   \end{proof}
% \item If $t > 1$ and $n > (b - 1)/(t - 1)$, then $b^{1/n} < t$.
%   \label{b-nth-root-less-than-t}
%   \begin{proof}
%     Suppose $t$ and $n$ satisfy the stated conditions. Part
%     \ref{b-nth-root-inequality} implies
%     \begin{align*}
%       n > \frac{b-1}{t-1} \geq \frac{n(b^{1/n} - 1)}{t-1}
%     \end{align*}
%     or
%     \begin{equation*}
%       1 > \frac{b^{1/n} - 1}{t-1}.
%     \end{equation*}
%     Since $t>1$, we can rearrange to get
%     \begin{equation*}
%       t > b^{1/n}
%     \end{equation*}
%     as desired.
%   \end{proof}
% \item If $w$ is such that $b^w<y$, then $b^{w+(1/n)}<y$ for
%   sufficiently large $n$.
%   \begin{proof}
%     \label{b-to-the-w-less-than-y}
%     Let $t = yb^{-w}$. Since $y > b^w$, we have $t > 1$. By the
%     archimedean property of $R$ (see Theorem~1.20a), there is a
%     positive integer $n$ such that
%     \begin{equation*}
%       n(t - 1) > b - 1.
%     \end{equation*}
%     So, by part \ref{b-nth-root-less-than-t}, we have
%     $b^{1/n} < t = yb^{-w}$ which gives
%     \begin{equation*}
%       b^{w+(1/n)} < y
%     \end{equation*}
%     as required.
%   \end{proof}
% \item If $b^w > y$, then $b^{w-(1/n)}>y$ for sufficiently large $n$.
%   \begin{proof}
%     \label{b-to-the-w-greater-than-y}
%     Let $t = b^w/y$. Then $t > 1$. Again, we can find an $n$ such that
%     $n(t-1) > b-1$ so part \ref{b-nth-root-less-than-t} gives
%     $b^{1/n} < b^w/y$ or, rearranging,
%     \begin{equation*}
%       b^{w-(1/n)} > y. \qedhere
%     \end{equation*}
%   \end{proof}
% \item Let $A$ be the set of all $w$ such that $b^w < y$, and show that
%   $x = \sup A$ satisfies $b^x = y$.
%   \begin{proof}
%     There are three possibilities: $b^x < y$, $b^x > y$, or $b^x = y$.

%     Suppose that $b^x < y$. Then by part \ref{b-to-the-w-less-than-y}
%     there is a positive integer $n$ such that $b^{x+1/n} <
%     y$. Therefore $x + 1/n\in A$ which contradicts the fact that $x$
%     is an upper bound for $A$.

%     On the other hand, if $b^x > y$, then part
%     \ref{b-to-the-w-greater-than-y} allows us to find a positive
%     integer $n$ such that $b^{x-1/n} > y$. Then $x-1/n\not\in A$ so
%     $x-1/n$ is an upper bound for $A$. But this contradicts the fact
%     that $x$ is the least upper bound for $A$.

%     We see that the only remaining possibility is that $b^x = y$.
%   \end{proof}
% \item Prove that this $x$ is unique.
%   \begin{proof}
%     Suppose $x$ is such that $b^x = y$ and let $z > x$. Then
%     \begin{equation*}
%       b^z = b^{x+(z-x)} = b^xb^{z-x} > b^x = y,
%     \end{equation*}
%     so $b^z\neq y$. Now suppose instead that $z < x$. If $b^z = y$
%     then the previous argument shows that $b^x\neq y$, which is a
%     contradiction. Therefore $b^z\neq y$ in this case as well. So $x$
%     is unique.
%   \end{proof}
% \end{enumerate}

\Exercise8 Prove that no order can be defined in the complex field
that turns it into an ordered field.
\begin{proof}
  Assume the contrary, so that the set of complex numbers is an
  ordered field with order $<$. Then by Proposition~1.18d,
  $-1 = i^2 > 0$. But then $0 = -1 + 1 > 0 + 1 = 1$. Hence $0 >
  1$. But again by Proposition~1.18d, $1 = 1^2 > 0$. This is a
  contradiction, so the complex numbers cannot be an ordered field.
\end{proof}

\Exercise9 Suppose $z = a + bi$, $w = c + di$. Define $z < w$ if
$a < c$, and also if $a = c$ but $b < d$. Prove that this turns the
set of all complex numbers into an ordered set. Does this ordered set
have the least-upper-bound property?
\begin{note}
  For clarity, we will use the symbol $\lessdot$ to represent the
  complex ordering defined above. For the rest of this exercise, the
  ordinary $<$ symbol will only denote the usual ordering on $R$.
\end{note}
\begin{proof}
  Let $z = a + bi$ and $w = c + di$ be arbitrary complex numbers, with
  $a,b,c,d\in R$. Since $R$ is an ordered field, exactly one of the
  statements $a < c$, $a > c$, or $a = c$ must be true. We consider
  each case in turn: First, if $a < c$ then $z \lessdot w$ but
  $w \nlessdot z$ and certainly $z\neq w$. Next, if $a > c$, then
  $w \lessdot z$ while $z \nlessdot w$ and $z\neq w$.

  In the case where $a = c$, then either $b < d$, $b > d$, or $b =
  d$. If $b < d$ then $z \lessdot w$ while $w \nlessdot z$ and
  $z\neq w$. Similarly, if $b > d$, then $w \lessdot z$ while
  $z \nlessdot w$ and $z\neq w$. And if $b = d$, then $z = w$ and
  neither of the statements $z \lessdot w$ and $w \lessdot z$ are
  true.

  In every case, exactly one of $z \lessdot w$, $w \lessdot z$, or
  $z = w$ is true.

  Lastly, suppose that $x = a_1 + b_1i$, $y = a_2 + b_2i$, and
  $z = a_3 + b_3i$ are complex numbers with $a_k,b_k\in R$ and such
  that $x \lessdot y$ and $y \lessdot z$. There are four cases: If
  $a_1 < a_2 < a_3$, or if $a_1 < a_2 = a_3$, or if $a_1 = a_2 < a_3$
  then $a_1 < a_3$ and $x \lessdot z$. The last case is where
  $a_1 = a_2 = a_3$. In that case we must have $b_1 < b_2 < b_3$ so
  that $b_1 < b_3$ and $x \lessdot z$. This completes the proof.
\end{proof}

\begin{claim}
  This ordered set does not have the least-upper-bound property.
\end{claim}
\begin{proof}
  Define $A$ to be the set of all complex numbers $a + bi$ with
  $a,b\in R$ such that $a < 0$. Then clearly $A$ is bounded above by
  $0$. Now suppose that $z = x + yi$ is any upper bound for $A$.

  First note that $x\geq0$. For, if not, we could choose a real number
  $x'$ such that $x < x' < 0$. But then $x'\in A$ and $z\lessdot x'$,
  which would give a contradiction.

  Now let $y'$ be any real number less than $y$. Since $x\geq0$, the
  complex number $x + y'i$ is an upper bound for $A$. But
  $x + y'i \lessdot z$, so $z$ cannot be the least upper bound for
  $A$. Since $z$ was arbitrary, this shows that the nonempty set $A$,
  which is bounded above, has no least upper bound.
\end{proof}

\Exercise{10} Suppose $z = a + bi$, $w = u + iv$, and
\begin{equation}
  \label{eq:a-b-complex}
  a = \left(\frac{\abs{w} + u}2\right)^{1/2}, \quad
  b = \left(\frac{\abs{w} - u}2\right)^{1/2}.
\end{equation}
Prove that $z^2 = w$ if $v\geq0$ and that $(\bar z)^2 = w$ if
$v\leq0$. Conclude that every complex number (with one exception!) has
two complex square roots.
\begin{proof}
  Direct computation gives
  \begin{align*}
    z^2 &= a^2 - b^2 + 2abi \\
        &= \frac{\abs{w} + u}2 - \frac{\abs{w} - u}2
          + 2\left(\frac{\abs{w} + u}2\right)^{1/2}
          \left(\frac{\abs{w} - u}2\right)^{1/2}i \\
        &= u + \left(\abs{w}^2 - u^2\right)^{1/2}i \\
        &= u + \left(v^2\right)^{1/2}i \\
        &= u + \abs{v}i.
  \end{align*}
  Therefore $z^2 = w$ if $v\geq0$ and $z^2 = \overline{w}$ if
  $v\leq0$. In the latter case, we have
  $(\bar z)^2 = \overline{(z^2)} = w$ (by Theorem~1.31b).

  Now, let $w = u + vi$ be any complex number, and define $a$ and $b$
  as in \eqref{eq:a-b-complex}. If $v > 0$, then $z = a + bi$ and
  $-z = -a - bi$ are distinct complex numbers such that $z^2 = w$ and
  $(-z)^2 = w$. On the other hand, if $v < 0$, then $\bar{z} = a - bi$
  and $-\bar{z} = -a + bi$ are distinct values with $(\bar{z})^2 = w$
  and $(-\bar{z})^2 = w$. And lastly, if $v = 0$ and $u\neq0$ then
  $y = \abs{u}^{1/2}$ and $-y = -\abs{u}^{1/2}$ are distinct with
  $y^2 = w$ and $(-y)^2 = w$. Therefore the only complex number that
  does not have two distinct complex square roots is $0$ itself.
\end{proof}

\Exercise{11} If $z$ is a complex number, prove that there exists an
$r\geq0$ and a complex number $w$ with $\abs{w} = 1$ such that
$z = rw$. Are $w$ and $r$ always uniquely determined by $z$?
\begin{proof}
  If $z = 0$, we may take $r = 0$ and $w = 1$.

  Otherwise, $\abs{z} > 0$ and we may simply let
  \begin{equation*}
    r = \abs{z} \quad\text{and}\quad w = \frac{z}{\abs{z}}.
  \end{equation*}
  Then $z = rw$, and $r\geq0$ by Theorem~1.33a. Moreover,
  \begin{equation*}
    \abs{w} = (w\overline{w})^{1/2}
    = \left(\frac{z\bar{z}}{\abs{z}^2}\right)^{1/2}
    = 1. \qedhere
  \end{equation*}
\end{proof}
\begin{claim}
  $w$ and $r$ are uniquely determined by $z$ if and only if $z\neq0$.
\end{claim}
\begin{proof}
  First, fix a nonzero complex number $z$. Suppose
  $z = r_1w_1 = r_2w_2$, where $r_1,r_2\geq0$ and
  $\abs{w_1} = \abs{w_2} = 1$. Then
  \begin{equation*}
    r_1 = \abs{r_1}\abs{w_1} = \abs{z} = \abs{r_2}\abs{w_2} = r_2
  \end{equation*}
  and, since $z$ is nonzero, $r_1$ and $r_2$ are positive and we have
  \begin{equation*}
    w_1 = \frac{r_2w_2}{r_1} = w_2.
  \end{equation*}
  Therefore $w$ and $r$ are uniquely determined.

  To prove the other direction, suppose $z = 0$. Then we may let
  $r = 0$, $w_1 = 1$, and $w_2 = -1$ so that $z = rw_1 = rw_2$ but
  $w_1\neq w_2$.
\end{proof}

\Exercise{12} If $z_1,\ldots,z_n$ are complex, prove that
\begin{equation}
  \label{eq:gen-complex-triangle-ineq}
  \abs{z_1 + z_2 + \cdots + z_n}
  \leq \abs{z_1} + \abs{z_2} + \cdots + \abs{z_n}.
\end{equation}
\begin{proof}
  We use induction on $n$. The case where $n = 1$ is trivial. Now
  suppose \eqref{eq:gen-complex-triangle-ineq} holds for $n = k$ where
  $k$ is any positive integer. Then by Theorem~1.33e and the induction
  hypothesis we have
  \begin{align*}
    \abs{z_1 + z_2 + \cdots + z_k + z_{k+1}}
    &= \abs{z_1 + z_2 + \cdots + z_{k-1} + (z_k + z_{k+1})} \\
    &\leq \abs{z_1} + \abs{z_2} + \cdots + \abs{z_{k-1}}
      + \abs{z_k + z_{k+1}} \\
    &\leq \abs{z_1} + \abs{z_2} + \cdots + \abs{z_{k-1}}
      + \abs{z_k} + \abs{z_{k+1}}.
  \end{align*}
  Therefore \eqref{eq:gen-complex-triangle-ineq} holds for all
  positive integers $n$.
\end{proof}

\Exercise{13} If $x, y$ are complex, prove that
\begin{equation*}
  \big\lvert\abs{x} - \abs{y}\big\rvert \leq \abs{x - y}.
\end{equation*}
\begin{proof}
  The triangle inequality from Theorem~1.33e gives
  \begin{equation*}
    \abs{x} = \abs{x - y + y} \leq \abs{x - y} + \abs{y}
  \end{equation*}
  so $\abs{x} - \abs{y} \leq \abs{x - y}$. Similarly,
  \begin{equation*}
    \abs{y} = \abs{y - x + x} \leq \abs{x - y} + \abs{x}
  \end{equation*}
  which gives $\abs{y} - \abs{x} \leq \abs{x - y}$.

  Now there are two cases:
  \begin{align*}
    \big\lvert\abs{x} - \abs{y}\big\rvert = \abs{x} - \abs{y}
    \quad\text{if $\abs{x} - \abs{y} \geq 0$,}
  \end{align*}
  or
  \begin{align*}
    \big\lvert\abs{x} - \abs{y}\big\rvert = \abs{y} - \abs{x}
    \quad\text{if $\abs{x} - \abs{y} \leq 0$.}
  \end{align*}
  Either way, we get
  $\big\lvert\abs{x} - \abs{y}\big\rvert \leq \abs{x - y}$.
\end{proof}

\Exercise{14} If $z$ is a complex number such that $\abs{z} = 1$, that
is, such that $z\bar{z} = 1$, compute
\begin{equation*}
  \abs{1 + z}^2 + \abs{1 - z}^2.
\end{equation*}
\begin{solution}
  Using Theorem~1.31, we get
  \begin{align*}
    \abs{1 + z}^2 &= (1 + z)\overline{(1 + z)} \\
                  &= (1 + z)(1 + \bar{z}) \\
                  &= 1 + z + \bar{z} + z\bar{z} \\
                  &= 2 + 2\Realpart z,
  \end{align*}
  and, similarly,
  \begin{align*}
    \abs{1 - z}^2 &= (1 - z)(1 - \bar{z}) \\
                  &= 1 - z - \bar{z} + z\bar{z} \\
                  &= 2 - 2\Realpart z.
  \end{align*}
  Hence
  \begin{equation*}
    \abs{1 + z}^2 + \abs{1 - z}^2 = 4. \qedhere
  \end{equation*}
\end{solution}

\Exercise{15} Under what conditions does equality hold in the Schwartz
inequality?
\begin{solution}
  Let $A$, $B$, and $C$ be defined as in the proof of
  Theorem~1.35. From that proof, we have equality when $Ba_j = Cb_j$
  for each $j$ from $1$ to $n$. This will be the case when
  $a_j = kb_j$ for some constant $k = C/B$.
\end{solution}
