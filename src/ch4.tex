\chapter{Continuity}

\Exercise1 Suppose $f$ is a real function defined on $R^1$ which
satisfies
\begin{equation}
  \label{eq:diff-f-x-plus-h-and-f-x-minus-h}
  \lim_{h\to0}[f(x+h)-f(x-h)]=0
\end{equation}
for every $x\in R^1$. Does this imply that $f$ is continuous?
\begin{solution}
  No. Consider the function $f\colon R^1\to R^1$ given by
  \begin{equation*}
    f(x) =
    \begin{cases}
      1, & \text{if $x$ is an integer}, \\
      0, & \text{otherwise}.
    \end{cases}
  \end{equation*}
  This function clearly has infinitely many discontinuities.

  Now fix a particular real number $x$. If $x$ is an integer, then for
  any $\varepsilon > 0$ we may let $\delta = 1/2$, so that for all $h$
  with $0<\abs{h}<\delta$ we have
  \begin{equation*}
    \abs{f(x + h) - f(x - h)} = \abs{0 - 0} = 0 < \varepsilon.
  \end{equation*}
  On the other hand, if $x$ is not an integer, then let $n$ be the
  integer nearest to $x$. For any $\varepsilon > 0$ take
  \begin{equation*}
    \delta = \frac{\abs{x - n}}2.
  \end{equation*}
  Again, for all $h$ with $0<\abs{h}<\delta$ we have
  \begin{equation*}
    \abs{f(x + h) - f(x - h)} = \abs{0 - 0} = 0 < \varepsilon.
  \end{equation*}

  We have shown that the function $f$ satisfies
  \eqref{eq:diff-f-x-plus-h-and-f-x-minus-h} for any $x$ in $R^1$, but
  $f$ is not continuous.
\end{solution}

\Exercise2
\label{exercise:continuity:f-of-closure-is-subset-of-closure-of-f}
If $f$ is a continuous mapping of a metric space $X$ into a metric
space $Y$, prove that
\begin{equation*}
  f(\overline{E}) \subset \overline{f(E)}
\end{equation*}
for every set $E\subset X$. ($\overline{E}$ denotes the closure of
$E$.) Show, by an example, that $f(\overline{E})$ can be a proper
subset of $\overline{f(E)}$.
\begin{proof}
  Suppose $p\in f(\overline{E})$. If $p\in f(E)$ then certainly
  $p\in\overline{f(E)}$ and we are done. So suppose that
  $p\not\in f(E)$. We want to show that $p$ must be a limit point of
  $f(E)$.

  By the choice of $p$ there must be $q\in\overline{E}$ such that
  $p = f(q)$. But if $p\not\in f(E)$ then $q\not\in E$. Therefore $q$
  is a limit point of $E$.

  Take any neighborhood $N$ of $p$ having radius $r>0$. Since $f$ is
  continuous, we may find $\delta>0$ so that for any $x\in X$,
  \begin{equation*}
    d(x,q) < \delta
    \quad\text{implies}\quad
    d(f(x),p)<r.
  \end{equation*}
  And $q$ is a limit point of $E$, so we can certainly find $x\in E$
  such that $d(x,q)<\delta$. Then $f(x)\in f(E)$ by definition, and we
  have $f(x)\neq p$ and $f(x)\in N$. Therefore $p$ is indeed a limit
  point of $f(E)$ and the main proof is complete.

  Finally, to see that $f(\overline{E})$ can be a proper subset of
  $\overline{f(E)}$, consider the continuous function $f\colon\R\to\R$
  given by
  \begin{equation*}
    f(x) = \frac2{x^2 + 1}.
  \end{equation*}
  If $E = (1,\infty)$ then $f(E) = (0,1)$ so that
  $\overline{f(E)} = [0,1]$. However $\overline{E} = [1,\infty)$ so
  $f(\overline{E}) = (0,1]$. Then
  $f(\overline{E})\subset\overline{f(E)}$ but equality does not hold.
\end{proof}

\Exercise3 Let $f$ be a continuous real function on a metric space
$X$. Let $Z(f)$ (the {\em zero set} of $f$) be the set of all $p\in X$
at which $f(p) = 0$. Prove that $Z(f)$ is closed.
\begin{proof}
  Since the set $\{0\}$ is closed in $R^1$ and $f$ is continuous, the
  corollary to Theorem~4.8 guarantees that $f^{-1}(\{0\})$ is closed
  in $X$. But $Z(f) = f^{-1}(\{0\})$ by definition, so $Z(f)$ is
  closed.
\end{proof}

\Exercise4 Let $f$ and $g$ be continuous mappings of a metric space
$X$ into a metric space $Y$, and let $E$ be a dense subset of
$X$. Prove that $f(E)$ is dense in $f(X)$. If $g(p) = f(p)$ for all
$p\in E$, prove that $g(p) = f(p)$ for all $p\in X$. (In other words,
a continuous mapping is determined by its values on a dense subset of
its domain).
\begin{proof}
  Let $p\in f(X)$. Since $X = \overline{E}$, we have by
  Exercise~\ref{exercise:continuity:f-of-closure-is-subset-of-closure-of-f}
  that $p\in\overline{f(E)}$ so that $p$ is either in $f(E)$ or is a
  limit point of $f(E)$. Either way this shows that $f(E)$ is dense in
  $f(X)$.

  Next, suppose $g(p) = f(p)$ for all $p\in E$ and let $q\in X -
  E$. Since $E$ is dense in $X$, $q$ is a limit point of $E$. Let
  $\varepsilon>0$ be arbitrary. The continuity of $f$ allows us to
  then find $\delta_1>0$ such that, for $x\in X$,
  \begin{equation*}
    d(x,q) < \delta_1
    \quad\text{implies}\quad
    d(f(x),f(q)) < \frac\varepsilon2.
  \end{equation*}
  Similarly, the continuity of $g$ allows us to find $\delta_2>0$ such
  that
  \begin{equation*}
    d(x,q) < \delta_2
    \quad\text{implies}\quad
    d(g(x),g(q)) < \frac\varepsilon2.
  \end{equation*}
  Set $\delta = \min(\delta_1,\delta_2)$.

  Pick $r\in E$ such that $d(q,r) < \delta$ (this is possible since
  $q$ is a limit point of $E$). Since $g(r) = f(r)$, we then have
  \begin{align*}
    d(f(q),g(q)) &\leq d(f(q),f(r)) + d(f(r),g(q)) \\
                 &= d(f(q),f(r)) + d(g(r),g(q)) \\
                 &< \frac\varepsilon2 + \frac\varepsilon2
                   = \varepsilon.
  \end{align*}
  Since $\varepsilon$ was arbitrary, we must have $d(f(q),g(q)) = 0$
  so that $f(q) = g(q)$.
\end{proof}
