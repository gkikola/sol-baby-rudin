\chapter{Continuity}

\Exercise1 Suppose $f$ is a real function defined on $R^1$ which
satisfies
\begin{equation}
  \label{eq:diff-f-x-plus-h-and-f-x-minus-h}
  \lim_{h\to0}[f(x+h)-f(x-h)]=0
\end{equation}
for every $x\in R^1$. Does this imply that $f$ is continuous?
\begin{solution}
  No. Consider the function $f\colon R^1\to R^1$ given by
  \begin{equation*}
    f(x) =
    \begin{cases}
      1, & \text{if $x$ is an integer}, \\
      0, & \text{otherwise}.
    \end{cases}
  \end{equation*}
  This function clearly has infinitely many discontinuities.

  Now fix a particular real number $x$. If $x$ is an integer, then for
  any $\varepsilon > 0$ we may let $\delta = 1/2$, so that for all $h$
  with $0<\abs{h}<\delta$ we have
  \begin{equation*}
    \abs{f(x + h) - f(x - h)} = \abs{0 - 0} = 0 < \varepsilon.
  \end{equation*}
  On the other hand, if $x$ is not an integer, then let $n$ be the
  integer nearest to $x$. For any $\varepsilon > 0$ take
  \begin{equation*}
    \delta = \frac{\abs{x - n}}2.
  \end{equation*}
  Again, for all $h$ with $0<\abs{h}<\delta$ we have
  \begin{equation*}
    \abs{f(x + h) - f(x - h)} = \abs{0 - 0} = 0 < \varepsilon.
  \end{equation*}

  We have shown that the function $f$ satisfies
  \eqref{eq:diff-f-x-plus-h-and-f-x-minus-h} for any $x$ in $R^1$, but
  $f$ is not continuous.
\end{solution}

\Exercise2 If $f$ is a continuous mapping of a metric space $X$ into a
metric space $Y$, prove that
\begin{equation*}
  f(\overline{E}) \subset \overline{f(E)}
\end{equation*}
for every set $E\subset X$. ($\overline{E}$ denotes the closure of
$E$.) Show, by an example, that $f(\overline{E})$ can be a proper
subset of $\overline{f(E)}$.
\begin{proof}
  Suppose $p\in f(\overline{E})$. If $p\in f(E)$ then certainly
  $p\in\overline{f(E)}$ and we are done. So suppose that
  $p\not\in f(E)$. We want to show that $p$ must be a limit point of
  $f(E)$.

  By the choice of $p$ there must be $q\in\overline{E}$ such that
  $p = f(q)$. But if $p\not\in f(E)$ then $q\not\in E$. Therefore $q$
  is a limit point of $E$.

  Take any neighborhood $N$ of $p$ having radius $r>0$. Since $f$ is
  continuous, we may find $\delta>0$ so that for any $x\in X$,
  \begin{equation*}
    d(x,q) < \delta
    \quad\text{implies}\quad
    d(f(x),p)<r.
  \end{equation*}
  And $q$ is a limit point of $E$, so we can certainly find $x\in E$
  such that $d(x,q)<\delta$. Then $f(x)\in f(E)$ by definition, and we
  have $f(x)\neq p$ and $f(x)\in N$. Therefore $p$ is indeed a limit
  point of $f(E)$ and the main proof is complete.

  Finally, to see that $f(\overline{E})$ can be a proper subset of
  $\overline{f(E)}$, consider the continuous function $f\colon\R\to\R$
  given by
  \begin{equation*}
    f(x) = \frac2{x^2 + 1}.
  \end{equation*}
  If $E = (1,\infty)$ then $f(E) = (0,1)$ so that
  $\overline{f(E)} = [0,1]$. However $\overline{E} = [1,\infty)$ so
  $f(\overline{E}) = (0,1]$. Then
  $f(\overline{E})\subset\overline{f(E)}$ but equality does not hold.
\end{proof}
