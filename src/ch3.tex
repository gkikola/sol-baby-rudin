\chapter{Numerical Sequences and Series}

\Exercise1 Prove that convergence of $\{s_n\}$ implies convergence of
$\{\abs{s_n}\}$. Is the converse true?
\begin{proof}
  Suppose $\{s_n\}$ converges to $s$ for some complex sequence
  $\{s_n\}$ and $s\in C$. Let $\varepsilon > 0$ be arbitrary. Then we
  may find $N$ such that $\abs{s_n - s} < \varepsilon$ for all
  $n\geq N$. Then, by Exercise~\ref{exercise-abs-abs-x-minus-abs-y} we
  have
  \begin{equation*}
    \abs{\abs{s_n} - \abs{s}} \leq \abs{s_n - s} < \varepsilon
    \quad\text{for each $n\geq N$.}
  \end{equation*}
  Hence $\{\abs{s_n}\}$ converges to $\abs{s}$.

  Note that the converse is {\em not} necessarily true. For example
  the real sequence $\{a_n\}$ given by $a_n = (-1)^n$ does not
  converge even though $\{\abs{a_n}\}$ converges to $1$.
\end{proof}

\Exercise2 Calculate $\displaystyle\lim_{n\to\infty}(\sqrt{n^2+n}-n)$.
\begin{solution}
  We have
  \begin{align*}
    \lim_{n\to\infty}(\sqrt{n^2+n}-n)
    &= \lim_{n\to\infty}\frac{(\sqrt{n^2+n}-n)(\sqrt{n^2+n}+n)}
      {\sqrt{n^2+n}+n} \\[3pt]
    &= \lim_{n\to\infty}\frac{n}{\sqrt{n^2+n}+n} \\[3pt]
    &= \lim_{n\to\infty}\frac1{\sqrt{1+\frac{1}n}+1} \\
    &= \frac12. \qedhere
  \end{align*}
\end{solution}
