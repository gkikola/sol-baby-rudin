\chapter{Numerical Sequences and Series}

\Exercise1 Prove that convergence of $\{s_n\}$ implies convergence of
$\{\abs{s_n}\}$. Is the converse true?
\begin{proof}
  Suppose $\{s_n\}$ converges to $s$ for some complex sequence
  $\{s_n\}$ and $s\in C$. Let $\varepsilon > 0$ be arbitrary. Then we
  may find $N$ such that $\abs{s_n - s} < \varepsilon$ for all
  $n\geq N$. Then, by Exercise~\ref{exercise-abs-abs-x-minus-abs-y} we
  have
  \begin{equation*}
    \abs{\abs{s_n} - \abs{s}} \leq \abs{s_n - s} < \varepsilon
    \quad\text{for each $n\geq N$.}
  \end{equation*}
  Hence $\{\abs{s_n}\}$ converges to $\abs{s}$.

  Note that the converse is {\em not} necessarily true. For example
  the real sequence $\{a_n\}$ given by $a_n = (-1)^n$ does not
  converge even though $\{\abs{a_n}\}$ converges to $1$.
\end{proof}

\Exercise2 Calculate $\displaystyle\lim_{n\to\infty}(\sqrt{n^2+n}-n)$.
\begin{solution}
  We have
  \begin{align*}
    \lim_{n\to\infty}(\sqrt{n^2+n}-n)
    &= \lim_{n\to\infty}\frac{(\sqrt{n^2+n}-n)(\sqrt{n^2+n}+n)}
      {\sqrt{n^2+n}+n} \\[3pt]
    &= \lim_{n\to\infty}\frac{n}{\sqrt{n^2+n}+n} \\[3pt]
    &= \lim_{n\to\infty}\frac1{\sqrt{1+\frac{1}n}+1} \\
    &= \frac12. \qedhere
  \end{align*}
\end{solution}

\Exercise3 If $s_1 = \sqrt2$, and
\begin{equation*}
  s_{n+1} = \sqrt{2 + \sqrt{s_n}}
  \qquad\text{($n=1,2,3,\dots$),}
\end{equation*}
prove that $\{s_n\}$ converges, and that $s_n<2$ for
$n = 1,2,3,\dots$.
\begin{proof}
  We will show by induction on $n$ that $\{s_n\}$ is a strictly
  increasing sequence that is bounded above by $2$. Certainly
  $\sqrt2 < \sqrt{2 + \sqrt2} < 2$, so the base case is
  satisfied. Suppose $s_n < s_{n+1} < 2$ for a positive integer
  $n$. Then
  \begin{equation*}
    s_{n+2} = \sqrt{2 + \sqrt{s_{n+1}}} > \sqrt{2 + \sqrt{s_n}} = s_{n+1}
  \end{equation*}
  and
  \begin{equation*}
    s_{n+2} = \sqrt{2 + \sqrt{s_{n+1}}}
    < \sqrt{2 + \sqrt{2}} < \sqrt{4} = 2.
  \end{equation*}
  Therefore $s_{n+1} < s_{n+2} < 2$ and it follows that $\{s_n\}$ is
  monotonic and bounded, and hence must converge.
\end{proof}

\Exercise4 Find the upper and lower limits of the sequence $\{s_n\}$
defined by
\begin{equation*}
  s_1 = 0;\quad s_{2m} = \frac{s_{2m-1}}2;
  \quad s_{2m+1} = \frac12 + s_{2m}.
\end{equation*}
\begin{solution}
  $\{s_n\}$ is the sequence
  \begin{equation*}
    0, \frac12, \frac14, \frac34, \frac38, \frac78,
    \frac7{16}, \frac{15}{16}, \dots.
  \end{equation*}
  The odd terms of $\{s_n\}$ form the sequence
  \begin{equation*}
    0, \frac14, \frac38, \frac7{16}, \dots, \frac{2^{n-1}-1}{2^n}, \dots
  \end{equation*}
  while the even terms form the sequence
  \begin{equation*}
    \frac12, \frac34, \frac78, \frac{15}{16}, \dots,
    \frac{2^n-1}{2^n}, \dots.
  \end{equation*}
  So
  \begin{align*}
    \liminf_{n\to\infty}s_n
    &= \lim_{n\to\infty}\frac{2^{n-1} - 1}{2^n} \\
    &= \lim_{n\to\infty}\left(\frac12 - \frac1{2^n}\right) \\
    &= \frac12,
  \end{align*}
  and
  \begin{align*}
    \limsup_{n\to\infty}s_n
    &= \lim_{n\to\infty}\frac{2^{n} - 1}{2^n} \\
    &= \lim_{n\to\infty}\left(1 - \frac1{2^n}\right) \\
    &= 1. \qedhere
  \end{align*}
\end{solution}

\Exercise5 For any two real sequences $\{a_n\}$, $\{b_n\}$, prove that
\begin{equation*}
  \limsup_{n\to\infty}(a_n + b_n)
  \leq \limsup_{n\to\infty}a_n + \limsup_{n\to\infty}b_n,
\end{equation*}
provided the sum on the right is not of the form $\infty - \infty$.
\begin{proof}
  For each positive integer $n$, put $c_n = a_n + b_n$. Let
  \begin{equation*}
    \alpha = \limsup_{n\to\infty}a_n,
    \quad
    \beta = \limsup_{n\to\infty}b_n,
    \quad\text{and}\quad
    \gamma = \limsup_{n\to\infty}c_n.
  \end{equation*}
  If $\alpha = \infty$ and $\beta\neq-\infty$ then the result is
  clear, and the case where $\alpha = -\infty$ and $\beta\neq\infty$
  is similar.

  So suppose $\alpha$ and $\beta$ are both finite. Let $\{c_{n_i}\}$
  be a subsequence of $\{c_n\}$ that converges to $\gamma$. Now let
  $\{a_{n_{i_j}}\}$ be a subsequence of $\{a_{n_i}\}$ such that
  \begin{equation*}
    \lim_{j\to\infty}a_{n_{i_j}} = \limsup_{i\to\infty}a_{n_i}.
  \end{equation*}
  Now since $\{c_{n_{i_j}}\}$ is a subsequence of $\{c_{n_i}\}$, it
  converges to the same limit $\gamma$. Then
  \begin{equation*}
    \lim_{j\to\infty}b_{n_{i_j}} = \lim_{j\to\infty}(c_{n_{i_j}} - a_{n_{i_j}})
    = \lim_{j\to\infty}c_{n_{i_j}} - \lim_{j\to\infty}a_{n_{i_j}}
    = \gamma - \limsup_{i\to\infty}a_{n_i}.
  \end{equation*}
  Rearranging, we get
  \begin{equation*}
    \gamma = \limsup_{i\to\infty}a_{n_i} + \lim_{j\to\infty}b_{n_{i_j}}.
  \end{equation*}
  But
  \begin{equation*}
    \limsup_{i\to\infty}a_{n_i} \leq \alpha
    \quad\text{and}\quad
    \lim_{j\to\infty}b_{n_{i_j}} \leq \beta,
  \end{equation*}
  so
  \begin{equation*}
    \gamma \leq \alpha + \beta
  \end{equation*}
  and the proof is complete.
\end{proof}

\Exercise6 Investigate the behavior (convergence or divergence) of
$\sum a_n$ if
\begin{enumerate}
\item $a_n = \sqrt{n+1} - \sqrt{n}$
  \begin{solution}
    Let $s_n$ denote the $n$th partial sum of $\sum a_n$. A simple
    induction argument will show that $s_n = \sqrt{n+1} -
    \sqrt1$. Since $s_n\to\infty$ as $n\to\infty$, the series
    $\sum a_n$ diverges.
  \end{solution}
\item $\displaystyle a_n = \frac{\sqrt{n+1} - \sqrt{n}}n$
  \begin{solution}
    We have
    \begin{equation*}
      a_n = \frac{(\sqrt{n+1} - \sqrt{n})(\sqrt{n+1} + \sqrt{n})}
            {n(\sqrt{n+1} + \sqrt{n})}
      = \frac1{n\sqrt{n+1} + n\sqrt{n}}.
    \end{equation*}
    so
    \begin{equation*}
      a_n \leq \frac1{2n^{3/2}} < \frac1{n^{3/2}}.
    \end{equation*}
    So by comparison (Theorem~3.25) with the convergent series
    $\sum1/n^{3/2}$, we see that $\sum a_n$ converges.
  \end{solution}
\item $a_n = (\sqrt[n]{n} - 1)^n$
  \begin{solution}
    By Theorem~3.20 (c),
    \begin{equation*}
      \lim_{n\to\infty}\sqrt[n]{a_n} = \lim_{n\to\infty}(\sqrt[n]{n} - 1)
      = 1 - 1 = 0.
    \end{equation*}
    Therefore, by the root test (Theorem~3.33), the series $\sum a_n$
    converges.
  \end{solution}
\item $\displaystyle a_n = \frac1{1+z^n}$ for complex values of $z$
  \begin{solution}
    First note that, by Exercise~\ref{exercise-abs-abs-x-minus-abs-y},
    we have
    \begin{equation*}
      \abs{z^n + 1} = \abs{z^n - (-1)}
      \geq \abs{\abs{z^n} - \abs{-1}}
      = \abs{\abs{z}^n - 1}.
    \end{equation*}
    Then
    \begin{equation}
      \label{eq:abs-1-over-1-plus-z-n-inequality}
      \Abs{\frac1{1 + z^n}} \leq \frac1{\abs{\abs{z}^n - 1}}.
    \end{equation}

    Now suppose $\abs{z} > 1$. Then there is an integer $N$ such that
    $\abs{z}^n > 2$ for all $n\geq N$. That is,
    \begin{equation*}
      \frac1{\abs{z}^n - 1}\leq \frac2{\abs{z}^n}
      \quad\text{for $n\geq N$}.
    \end{equation*}
    Using this fact, \eqref{eq:abs-1-over-1-plus-z-n-inequality}
    becomes
    \begin{equation*}
      \Abs{\frac1{1 + z^n}} \leq \frac2{\abs{z}^n}
      \quad
      \text{for $n\geq N$}.
    \end{equation*}
    So by the comparison test with the convergent geometric series
    $\sum 2/\abs{z}^n$ we have that $\sum a_n$ also converges.

    In the case where $\abs{z} \leq 1$, it is easy to see that
    $a_n\not\to 0$ as $n\to\infty$, so $\sum a_n$ diverges.
  \end{solution}
\end{enumerate}
